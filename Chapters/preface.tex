\chapter*{About this template}
\addcontentsline{toc}{chapter}{About this template} % DO NOT CHANGE 
\label{chap:about}


This document, along with the source code, is meant to be a self-explanatory generic thesis template, to be used in any field. It is implemented in \LaTeX, the most popular, advanced, flexible, and comprehensive documentation system in natural sciences. However, the document as such could be implemented with other tools, as well.

The thesis covers both structure, content, and layout. No prior \LaTeX\ knowledge is required, since much emphasis has been put on making it user friendly and fairly tampering proof.

The suggested structure is based on the widely used \index{IMRAD} model (Introduction, Methods, Results, And Discussion)\footnote{\url{en.wikipedia.org/wiki/IMRAD}}. The student may of course deviate from the structure and recommended content for each chapter. In particular, the chapters describing the main bulk of work done in the research project (Chapters \ref{chap:design}, \ref{chap:implementation}, and \ref{chap:evaluation}), should be customized to fit the specific topic of your project, both regarding chapter titles and content. You may, of course, consider merging some of the chapters, and/or add more chapters.

The template is based on the author's personal experiences as a research scientist and lecturer during the last 25 years\footnote{\url{www.ia.hiof.no/~gunnarmi}}, various online resources, 
and the ``The Mayfield Handbook of Technical and Scientific Writing"  \parencite{perelman97mht}\footnote{\url{www.mhhe.com/mayfieldpub/tsw/home.htm}}.

For technical details, see Appendix \ref{chap:how-to}.

For updates and advice regarding this template, and similar templates for Microsoft Word and LibreOffice Writer, follow the Thesis Templates Facebook group\footnote{\url{https://www.facebook.com/groups/693636154649756}}.

Finally: Comments, bug reports, and suggestions are highly welcome\footnote{\url{gunnar.misund@hiof.no}}!

\vspace{20mm}

%\raggedleft
Gunnar Misund

Halden, \today
%\justify





